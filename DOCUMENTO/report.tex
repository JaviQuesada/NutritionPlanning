\documentclass[%
    school=etsisi,%
    type=pfg,%
    degree=61IW,%
    authorsex=m,%
    directorsex=m,%
]{upm-report}

\addbibresource{references.bib}

\title{Planificación nutricional mediante algoritmos evolutivos}
\author{Javier Quesada Pajares}
\director{Cristian Ramírez Atencia}

\abstract{spanish}{
    
    Para el final.
    
    \textbf{Condensado no quiere decir incompleto}. Debe contener la información más destacable. Lo ideal es que ocupe entre media y una cara de un folio A4. Comenzará por el propósito y principales objetivos de la memoria. Luego hablaremos sobre los aspectos más destacables de la metodología empleada, seguido de los resultados obtenidos. Por último se presentarán las conclusiones de forma condensada.
    
    Debe tener un estilo claro y conciso, sin ambigüedades de ningún tipo. Además, al ser un resumen de todo el contenido, ni que decir tiene que deberá ser lo último que elaboraremos, y deberá mantener una absoluta fidelidad con el contenido de la memoria.
}
\keywords{spanish}{Cuatro o cinco; Expresiones clave; Que resuman; Nuestro proyecto o; Investigación}

\abstract{english}{
    This section must contain the summary that we have written before in Spanish, but in English, as well as the keywords.
}
\keywords{english}{Four or five; Key Expressions; Summarising; Our Project or; Research}

\acknowledgements{
    Aquí los agradecimientos que quieras dar. Y si no quieres, borras la entrada \texttt{\textbackslash acknowledgements} de \texttt{report.tex} y ya está.
}

\begin{document}

\include{frontmatter/glossary}

\chapter{Introducción}
\label{ch:introduccion}

\section{Contexto}
En los últimos años, la \gls{ia} se ha posicionado como una de las herramientas más útiles e interesantes de las que disponemos. El término \textit{<<inteligencia artificial>>} fue acuñado por primera vez por John McCarthy en la Conferencia de Darmouth de 1956~\cite{dartmouth1956}. Años después de que Alan Turing formulase la pregunta sobre si las máquinas podían pensar y plantease el famoso Test de Turing~\cite{turing1950}, varios científicos se reunieron con el objetivo de discutir acerca de la posiblidad de un artefacto de comportarse de manera inteligente. Se llegó a la conclusión de que todo aspecto del aprendizaje se puede describir con tanta precisión que resulte factible construir una máquina que los simule.

La IA se enfoca en crear sistemas que puedan realizar tareas que normalmente requerirían de inteligencia humana. Aunque ha tenido un reciente auge debido a los chatbots o los asistentes personales, presenta una gran cantidad de finalidades. Dentro de los usos que se le dan a la IA destacan la creación y el análisis de productos o la automatización de servicios, además de la optimización de procesos. Esta optimización se enfoca en encontrar la mejor solución posible a un problema dado dentro de un conjunto de opciones factibles. Los algoritmos de optimización y de personalización permiten, por ejemplo, crear aplicaciones que permiten organizar tareas de manera inteligente~\cite{todoist2024} o termostatos que ajustan la calefacción automaticamente~\cite{googlenest2020}.

Este \gls{pfg} se centrará en la resolución de un problema de optimización, la creación de un menú semanal de comidas personalizado que cumpla distintos objetivos, como el número de calorías diarias o la cantidad de macronutrientes ingeridos. Se hará uso de la computación evolutiva que, mediante algoritmos genéticos, diseñará una dieta equilibrada a partir de la selección, cruce y mutación de los distintos alimentos.


\section{Motivación}

La motivación para realizar este proyecto viene dada por el interés creciente en el área de la inteligencia artificial y cómo ha cambiado la manera de plantear los problemas respecto al pasado. En España, el sector TIC (Tecnologías de la Información y la Comunicación) cuenta con más del 40\% de empresas que usan estas herramientas para la automatización de flujos de trabajo, análisis de datos o para la gestión de la cadena de suministro~\cite{ontsi2023}. Buscan mejorar la precisión y la eficiencia a la hora de diseñar soluciones.

No obstante, no solo en sectores tecnológicos se hace uso de la IA. El médico o el alimentario también están incorporándola de manera gradual. Los diagnósticos de imágenes médicas~\cite{philips2024} o la agricultura de precisión~\cite{majeed2024} son cada vez más comunes. También en la nutrición, relacionada con esto ámbitos, estas tecnologías permiten nuevas posibilidades.

\begin{figure}[H]
    \centering
    \includegraphics[width=0.75\textwidth]{figures/prevalencia-obesidad.png}
    \caption{Prevalencia (\%) de obesidad y exceso de peso por grupos de sexo y edad. Fuente \cite{ENE-COVID}}
    \label{fig:prevalencia-obesidad}
\end{figure}

Según un estudio llevado a cabo por la Agencia Española de Seguridad Alimentaria y Nutrición (AESAN) y por el Instituto de Salud Carlos III (ISCIII), un 55,8\% de la población adulta española tiene exceso de peso y un 18,7\% padece obesidad, tal y como se muestra en la figura \ref{fig:prevalencia-obesidad}. Esto trae consigo múltiples problemas de salud, como la aparición de enfermedades crónicas o cardiovasculares, complicaciones respiratorias o dificultades al moverse.

Estos datos demuestran la importancia de una buena nutrición y hábitos alimenticios saludables. La planificación nutricional mediante algoritmos genéticos ayuda a comprender cómo se pueden mejorar estos hábitos a través de estas técnicas innovadoras, además de promover una vida sana y bienestar para todos, lo que se relaciona con los Objetivos de Desarrollo Sostenible (ODS)~\cite{ONU2024}, por lo que es un gran aliciente a la hora de realizar este PFG.


\section{Justificación}

En esta sección se deben explicar y argumentar las razones por las cuales se eligió el tema del proyecto, así como su importancia y relevancia. Algunos elementos clave que se pueden abordar en esta sección son:

\begin{enumerate}
    \item \textbf{Relevancia del tema}: ¿Existe alguna necesidad o problema específico que tu proyecto pueda abordar?
    \item \textbf{Justificación teórica}: Mención sobre qué teorías, enfoques o modelos existentes en la literatura respalden la importancia de abordar este tema.
    \item \textbf{Brecha en el conocimiento}: ¿Qué aspectos no se han explorado lo suficiente o no han sido abordados en estudios previos? ¿Cómo puede el proyecto contribuir a cerrar esa brecha en el conocimiento?
    \item \textbf{Contribución práctica}: Aplicaciones del proyecto y cómo pueden beneficiar a la comunidad académica, profesional o a la sociedad en general.
\end{enumerate}

La sección no tiene por qué ser demasiado extensa, ni tiene por qué incluir (o limitarse) a los puntos anteriores, pero debe ser lo suficientemente clara y convincente para que los lectores comprendan por qué el proyecto es relevante y necesario.


\section{Objetivos}

El objetivo general de este proyecto es la creación de un planificación semanal de comidas mediante algoritmos genéticos. Experimentando con diferentes algoritmos se busca crear un menú que cumpla con distintos objetivos nutricionales, como la ingesta calórica diaria, a la vez que limitarse según las restricciones dadas. Los objetivos específicos de este PFG son:

\begin{itemize}
    \item Desarrollar un algoritmo evolutivo capaz de generar menús que cumplan con las restricciones y los objetivos nutricionales establecidos.
    \item Personalizar el algoritmo para la variación de las comidas según las necesidades específicas de los individuos.
    \item Experimentar con distintas configuraciones y variantes del algoritmo genético en busca de encontrar la mejor solución posible.
    \item Evaluar la sensibilidad y eficacia del algoritmo.
    \item Ejecutar pruebas que validen el algoritmo.
    \item Documentar los resultados obtenidos.
\end{itemize}

\begin{comment}
Una de las partes más importante y complicada. Se considera \textbf{la finalidad} del proyecto en cuestión a realizar y suele encajar dentro de una de las siguientes categorías:

\begin{itemize}
    \item \textbf{Contraste} o validación de una hipótesis. Suele usarse en \glspl{pfm}, no tanto en \glspl{pfg}.
    \item \textbf{Desarrollo} o diseño de algo (e.g.~Software, hardware, sistema, edificio). Suele ser el más común en las ingenierías.
    \item \textbf{Estudio} de un tema que deduce o descubre nuevo conocimiento. Suele ser más común en las ramas de las ciencias puras y humanidades.
\end{itemize}

Sirve como primer indicador de la consecución del proyecto, ya que planteando objetivos podemos determinar en las conclusiones su grado de cumplimiento. Ahora bien, ¿cómo determinamos que un objetivo se ha cumplido? pues definiéndolo para que se pueda cumplir, es decir, intentando que sea:

\begin{itemize}
    \item \textbf{Acotado en el tiempo}, así es más fácil establecer un marco temporal para su realización y programar temporalmente las partes de las que se compone.
    \item \textbf{Medible}, para saber cómo de lejos estamos de llegar a un resultado aceptable.
    \item \textbf{Específico}, de manera que esté bien acotado y sea difícil embarcarse en tareas que no nos acerquen a su consecución.
    \item \textbf{Alcanzable}, porque si no lo es, por mucha intención y esfuerzo que le pongamos no se va a terminar.
    \item \textbf{Relevante}, porque si, en un \gls{pfg} para Ingeniería del Software, desarrollamos un producto mecánico para sexar pollos, pues por muy importante que sea, poco tiene que ver con lo que se ha estudiado durante todos estos años.
\end{itemize}

Regla mnemotécnica: \textit{AMEAR}.
\end{comment}

\section{Estructura de la memoria}

En este subapartado se explicará la estructura del documento.

En el capítulo 2, \nameref{ch:estado-arte}, se centra en dar un enfoque general al proyecto. Se da contexto sobre las áreas en las que se basa el proyecto, la inteligencia artificial y los algoritmos evolutivos, además de sobre el tema central del PFG, la planificación nutricional mediante algoritmos evolutivos.

En el capítulo 3, \nameref{ch:marco-teorico}, se aporta la base teórica que en la que se desarrolla el proyecto. Se da una explicación general sobre la inteligencia artificial y se indaga en el algoritmo evolutivo y su funcionamiento.

En el capítulo 4, \nameref{ch:desarrollo}, 
\chapter{Estado del arte}
\label{ch:estado-arte}

\section{Inteligencia artificial}

El campo de la inteligencia artificial en la actualidad se encuentra en un estado de rápida evolución, con una infinidad de usos que se extiende por una gran variedad de sectores económicos y sociales. Su capacidad para analizar y procesar grandes cantidades de datos o para mejorar la eficiencia en distintas industrias han hecho que se trate de una tecnología en auge. El estudio de McKinsey \& Company The state of AI in 2022 and a half decade in review estima que el 50\% de las empresas ya usan IA en sus tareas diarias, donde destacan la optimización de servicios, la creación de productos y el análisis del servicio al cliente.

El informe AI Index Report 2024 del Human-Centered AI Institute (HAI) de la Universidad de Stanford muestra que la IA ya supera a las habilidades humanas en ciertas tareas, como en la clasificación de imágenes, con una precisión del 97\% respecto al 95\% de los humanos, o en juegos, como el ajedrez, donde supera consistentemente a los jugadores humanos.

Aunque sin duda, el mayor crecimiento en los dos últimos años corresponde a las herramientas relacionas con la Inteligencia Artificial Generativa (IAG). Esta rama se centra en crear modelos capaces de generar contenido, como conversaciones, imágenes, videos o música. Aprende de datos ya existentes y produce nuevos con características similares. En este ámbito destaca la empresa OpenAI, que cuenta con ChatGPT, capaz de generar texto lógico y actuar como chatbot, o con Dall-E, que puede generar imágenes en base a la descripción que se entregue. The state of AI in 2023: Generative AI's breakout year, también de McKinsey \& Company, muestra que el 79\% de los encuestados dice haber tenido almenos alguna exposición a la IAG, lo que indica que es la herramienta de IA que más rápido está captando el interés del público general.

\begin{figure}[H]
    \centering
    \includegraphics[width=0.5\textwidth]{figures/dall-e.png}
    \caption{Imagen generada con DALL-E 3.}
    \label{fig:dall-e}
\end{figure}

Existe una gran cantidad de proyectos pioneros actualmente relacionados con el campo de la IA. Algunos de los más importantes actualmente son:

\begin{itemize}
    \item AlphaFold. Este programa de Deepmind utiliza IA para predecir las estructuras de las proteínas. Ha permitido acelerar las investigaciones científicas que desembocan en la creación de nuevos medicamentos.
    \item Neuralink. Esta empresa trabaja en el desarrollo de un chip que permita comunicar directamente un cerebro humano con una computadora. Posibilitará realizar análisis neurológicos más precisos.
    \item Vehículos autónomos. Waymo controla una flota de taxis sin conductor que han estado operando por Estados Unidos. 
    \item Medicina personalizada. Análisis de datos genéticos y médicos para diseñar tratatamientos que sean más efectivos para el paciente.
\end{itemize}

\newpage

\section{Algoritmos evolutivos}

Los algoritmos evolutivos han tenido varios proyectos que han marcado el desarrollo de esta tecnología. Los primeros acercamientos tuvieron lugar en la década de 1960. Por un lado, Lawrence Fogel comenzó a explorar la programación evolutiva, centrándose en la evolución de autómatas finitos. Uno de los primeros intentos de aplicar principios evolutivos a la informática. Por otro lado, Rechenberg y Shwefel desarrollaron estrategias de evolución para problemas de optimización en ingeniería. Centradas en la selección y en la mutación, resultaron ser muy útiles para aplicarlas en la realidad.

Tras estos primeros pasos, John Holland fue fundamental para el desarrollo de los algoritmos genéticos. Su libro Adaptation in Natural and Artificial Systems del año 1975 es básico para entender el funcionamiento de los mismos. Su trabajo se centró en la idea de que la evolución biológica podía ser simulada y utilizada para resolver problemas complejos en computación. Hacen uso de los operadores como la selección, el cruce o la mutación para evolucionar una población candidata de individuos hacia una mejor solución. Cada solución es evaludada seguún una función de fitness, que mide qué tan buena es la solución al problema en cuestión.

\begin{lstlisting}[caption=Algoritmo Genético]
INICIAR
    INICIALIZAR población
    EVALUAR fitness
    REPETIR HASTA CUMPLIR condición de parada:
        SELECCIONAR individuos
        CRUZAR padres
        MUTAR hijos
        EVALUAR individuos
        FORMAR nueva generación
    RETORNAR solución
FINALIZAR
\end{lstlisting}

Después de que Holland sentara las bases de los algoritmos genéticos, en los años siguientes fueron surgiendo estudios que hicieron evolucionar la rama. Varios de los más importantes fueron de John Koza, quien introdujo la programación genética, técnica usada para desarrollar automáticamente programas que realicen una tarea definida por el usuario. Se optimiza una población de individuos (programas) respecto a una función de aptitud. Koza probó su viabilidad para resolver problemas de robótica o de optimización.

Tras él han seguido surgiendo nuevas técnicas dentro de los algoritmos evolutivos. Entre ellas se puede destacar el desarrollo de los Algoritmos genéticos híbridos (AGH), que combinan los algoritmos genéticos con otras técnicas de optimización para mejorar las soluciones a los problemas complejos. Otro avance importante es el de la Programación genética cartesiana (CGP), que sustituye los árboles de busca usados en la programación genética tradicional por grafos dirigidos, lo que es muy útil, por ejemplo, en el diseño de circuitos electrónicos.

A lo largo de los años se han desarrollado distintos proyectos pioneros que han hecho uso de los algoritmos genéticos. Algunos son:

\begin{itemize}
    \item Space Technology 5 (ST5). Se usó algoritmos genéticos para la creación de una antena ultracompacta para la misión ST5 de la NASA, superando las expectativas de rendimiento.
    \item The EvoTanks Project. Se estudió el uso de algoritmos genéticos para desarrollar estrategias de combate para tanques autónomos en simulaciones.
    \item Sector financiero. Se puede utilizar para predecir movimientos del mercado. Además, a nievel doméstico, hay apps que lo usan para optimizar el método de compartir gastos entre distintos usuarios.
    \item Bioingeniería. Ayudan a modelar secuencias genéticas, que acelera los avances en medocina personalizada.
\end{itemize}


\section{Planificación nutricional mediante algoritmos evolutivos}

Los primeros acercamientos para intentar resolver problemas de optimización nutricional se dan en la década de 1940. George Stigler, en su artículo The Cost of subsistence, planteó el problema de encontrar la dieta de menor coste que cumpliese con unos objetivos nutricionales. Al no poseer ordenadores, utilizó técnicas manuales.

El problema fue formalmente resuelto dos años después por Jack Laderman usando programación lineal. Fue capaz de calcular la combinación de alimentos que cumpliese con los requisitos económicos y nutricionales. Con esto se demostró la utilidad de técnicas computacionales en tareas de optimización.

Tras las bases que sentó Holland, empezaron a aparecer distintos trabajos que hacían uso de los algoritmos genéticos para resolver problemas de optimización, incluyendo los de planificación nutricional. Application of genetic algorithms to diet optimization problems por 









\chapter{Marco teórico}
\label{ch:marco-teorico}

\section{Inteligencia artificial}

La \gls{ia} se puede definir como el estudio y diseño de agentes inteligentes, es decir, de sistemas que perciben su entorno y toman decisiones o acciones que maximizan sus posibilidades de éxito. Este campo se basa en disciplinas como la informática, la lógica o la neurociencia, que contribuyen a la simulación de capacidades cognitivas humanas en máquinas.

Los agentes inteligentes, que son la base de este campo, se clasifican según la capacidad de reconocer y actuar en su entorno. Podemos encontrar desde agentes simples, que responden directamente a estímulos, hasta agentes basados en utilidad, que evalúan si los resultados de sus acciones son satisfactorios. Es fundamental la racionalidad, la habilidad de realizar elecciones óptimas que maximicen la posibilidad de alcanzar objetivos. Utilizando distintas herramientas de lógica, los agentes inteligentes pueden formular y modificar conocimientos, deduciendo nueva información.

El razonamiento lógico es también fundamental para el desarrollo de algoritmos evolutivos, donde las decisiones sobre selección, cruzamiento o reproducción se basan en decisiones lógicas. Subclase de los métodos de aprendizaje automático inspirados en los procesos biológicos de evolución, aplican estos principios para desarrollar soluciones a complejos problemas.

A medida que la tecnología evoluciona, también lo hace la capacidad de la IA para aprender y adaptarse, al igual que los algoritmos genéticos mejoran a lo largo de las generaciones. Su implementación es cada vez mayor en campos como el desarrollo de software o la robótica, donde las mejoras en la optimización y en la resolución de los problemas conducen a mejoras significativas en la eficiencia y la funcionalidad.


\section{Algoritmos genéticos}

Un algoritmo se puede definir como un procedimiento computacional bien definido que toma un valor, o un conjunto de valores, como entrada y produce un valor, o un conjunto de valores, como salida. Se trata de un conjunto de instrucciones que permiten realizar una actividad mediante sucesivos pasos.

Un algoritmo evolutivo es un tipo de algoritmo que proviene de la computación evolutiva. Esta rama de la IA emplea principios inspirados en la evolución biológica para la resolución de problemas. Teoría propuesta por Charles Darwin en 1859, expone que en la naturaleza, en un entorno dado que puede albergar un número limitado de individuos, la selección natural favorecería a aquellos que poseyeran características que les permitiesen adaptarse mejor al medio, teniendo una mayor posibilidad de sobrevivir y reproducirse. A lo largo del tiempo, las características ventajosas se propagan a través de las generaciones, mientras que aquellas menos favorables tienden a desaparecer. Por lo tanto, estas poblaciones irán evolucionado y adaptándose gradualmente al entorno.

En 1975 John Holland propone imitar los procesos biológicos naturales que rigen la selección natural usando ordenadores, lo que sería el principio de los \glspl{ag}. En un AG, las soluciones son modeladas como individuos o cromosomas, que generalmente se representan como cadenas de bits, aunque también se puede usar otro tipo de cadenas. Estas soluciones son evaluadas por una función de aptitud o fitness, y las más adecuadas son selecciondas para reproducirse mediante cruzamiento y mutación, procesos que mezclan y alteran aleatoriamente los cromosomas para generar diversidad. Los descendientes resultantes forman nuevas generaciones que vuelven a ser evaluadas, creando un ciclo que se repite hasta cumplir alguna condición de parada, como pudiera ser un número limitado de generaciones o que se alcance la solución deseada.

\begin{figure}[H]
    \centering
    \includegraphics[width=1\textwidth]{figures/algoritmo-genetico.png}
    \caption{Algoritmo genético simple.}
    \label{fig:algoritmo-genetico}
  \end{figure}

Explicado el concepto general, se va a desglosar cada una de las fases de la figura \ref{fig:algoritmo-genetico}.

\subsection{Población}

Antes de explicar la generación de una población inicial, es necesario conocer algunos conceptos que son usados para representar y entender las soluciones. Son términos que son usados en la biología y en el campo de la genética. 

\begin{itemize}
    \item \textbf{Gen.} Es la unidad básica de información en un cromosoma. En una cadena de bits que representa una solución, cada bit puede considerarse un gen. En el caso que se trata en este PFG, el menú semanal, cada tipo de comida se podría considerar un gen. Por ejemplo, un gen sería bebida, plato principal o postre.
    \item \textbf{Alelo.} Es la forma específica o valor que puede tomar un gen. Siguiendo el ejemplo de bebida, posibles alelos serían agua, té o cerveza.
    \item \textbf{Cromosoma.} Es una colección de genes y representa una solución completa al problema de optimización. El cromosoma es la cadena de bits completa. En nuestro caso, la lista completa de alimentos seleccionados de un día determinado es el cromosoma. 
    \item \textbf{Fenotipo.} Es la manifestación real de la solución codificada. En el PFG sería cómo se preparan y sirven estos alimentos en la realidad.
\end{itemize}

La generación de una población inicial implica crear un conjunto de soluciones candidatas. Generalmente, los individuos son seleccionados aleatoriamente dentro de los límites definidos en cada problema, asegurando que todas las áreas del espacio de búsqueda puedan ser exploradas. También existen métodos alternativos de generación que aplican ciertas restricciones para formar soluciones iniciales más prometedoras. Tomando de ejemplo el menú semanal, el espacio de búsqueda comprendería la base de datos en la que aparecen todos los alimentos con sus respectivas calorias y macronutrientes. 

\begin{figure}[H]
  \centering
  \includegraphics[width=0.625\textwidth]{figures/cromosoma.png}
  \caption{Estructura de un cromosoma.}
  \label{fig:cromosoma}
\end{figure}


\subsection{Evaluación}

Se mide lo bueno que es un individuo para nuestros propósitos (la calidad del individuo). Por lo tanto, lo más importante es definir una función de fitness correcta y representativa del problema a evaluar.

Según vayan pasando generaciones, la población inicial irá evolucionando hacia poblaciones candidatas que presentarán una mejor aptitud. Si la población ha alcanzado el objetivo, la condición de parada se activará, convirtiendo el conjunto de individuos candidatos en la población solución. En caso de no alcanzarlo, seguirá evolucionando hacia otra población candidata distinta.

\subsection{Operadores}

Son los procesos que se aplican a poblaciones de individuos para desarrollar generaciones futuras. Sirven para la exploración del espacio de soluciones y para la mejora de las poblaciones a través de las generaciones. Se busca que el conjunto de individuos presente una alta diversidad, ya que si es baja se corre el riego de caer en mínimos locales, lo que no permitiría explorar el espacio de soluciones ampliamente.


\subsubsection{Selección}

El primero de los operadores básicos. Elige un individuo para reproducirlo. Si bien se puede seleccionar de manera equiprobable, existen métodos basados en la aptitud del individuo. Estos son algunos:

\begin{itemize}
  \item \textbf{Método estándar (rueda de la fortuna).} Asigna a cada individuo una probabilidad proporcional a su fitness, por lo que los más aptos tienen mayor probabilidad de reproducirse. Existe un derivado de este método en el que se normalizan las probabilidades, lo que favorece aún más a los adaptados.
  \item \textbf{Método del rango.} Se ordenan los individuos según su aptitud, y la probabilidad de selección se asigna según este ránking.
  \item \textbf{Selección por torneo.} Se escoge aleatoriamente un subconjunto de individuos de la población, y el mejor de este es elegido.
\end{itemize}

La selección puede incorporar un mecanismo de elitismo, que escoge los mejores individuos de una generación para que se mantengan en la siguiente. Esto ayuda a que la calidad del mejor individuo de una generación sea siempre igual o superior a su equivalente de la generación anterior, consiguiendo una progresión constante hacia la solución óptima, es decir, una mayor convergencia.

También se debe buscar un equilibrio con la diversidad. Si no se explora el espacio de búsqueda en amplitud, se podría caer en mínimos locales si la población es muy parecida entre sí, lo que no permitiría encontrar buenas soluciones.


\subsubsection{Cruce}

Se combina el material genético de dos individuos, padres, para producir descendencia, hijos. Es decir, se utiliza para intercambiar características de dos soluciones parentales con el objetivo de generar nuevas soluciones. Se aplica con probabilidad \(P_c\) . Al promover la mezcla de genes, ayuda a aumentar la diversidad y la convergencia, debido a la generación de nuevos descendientes con características deseables. Para problemas donde las soluciones están representadas como cadena de bits o enteros, se usan distintos métodos:

\begin{itemize}
  \item \textbf{Cruce de un punto.} Se selecciona aleatoriamente una posición en el cromosoma. Todo lo que está antes de este punto se intercambia con todo lo que está después en la otra cadena, y viceversa, para producir dos descendientes.
  \item \textbf{Cruce de dos puntos.} Similar al cruce de un punto, pero se seleccionan dos puntos de corte. Las cadenas de genes entre estos dos puntos se intercambian entre los dos padres.
  \item \textbf{Cruce uniforme.} Cada gen tiene una probabilidad igual de ser elegido de uno de los dos padres.
\end{itemize}

\subsubsection{Mutación}

Último de los operadores básicos. Se recorre toda la cadena, mutando cada gen con probabilidad \(P_m\) , es decir, eligiendo un nuevo valor mediante una elección equiprobable sobre el alfabeto. La mutación aumenta la diversidad y ayuda a no caer en mínimos locales. Algunos de los métodos usados son:

\begin{itemize}
  \item \textbf{Mutación uniforme.} Cada gen puede cambiar a otro valor con probabilidad \(P_m\).
  \item \textbf{Mutación por intercambio.} Dos genes aleatorios en el cromosoma son seleccionados y sus posiciones se intercambian. 
\end{itemize}

\begin{figure}[H]
  \centering
  \includegraphics[width=1\textwidth]{figures/operadores.png}
  \caption{Operadores.}
  \label{fig:operadores}
\end{figure}
\chapter{Desarrollo}
\label{ch:desarrollo}

\lstdefinestyle{sqlstyle}{ % Configuración global de listings
    language=SQL,
    keywordstyle=\color{myblue},
    commentstyle=\color{darkgreen},
    stringstyle=\color{myorange},
    basicstyle=\ttfamily,
    morekeywords={CREATE, DATABASE, USE, ALTER, TABLE, ADD, COLUMN, UPDATE, SET, CHAR, INT, PRIMARY, KEY, NOT, NULL, SUBSTRING, VARCHAR, AUTO_INCREMENT},
    literate={0}{{{\color{myorange}0}}}{1}
             {1}{{{\color{myorange}1}}}{1}
             {2}{{{\color{myorange}2}}}{1}
             {3}{{{\color{myorange}3}}}{1}
             {4}{{{\color{myorange}4}}}{1}
             {5}{{{\color{myorange}5}}}{1}
             {6}{{{\color{myorange}6}}}{1}
             {7}{{{\color{myorange}7}}}{1}
             {8}{{{\color{myorange}8}}}{1}
             {9}{{{\color{myorange}9}}}{1},
}

\section{Base de datos}

\subsection{Selección de la base de datos}
La base de datos de alimentos y platos utilizada para el desarrollo del algoritmo es la perteneciente al gobierno del Reino Unido. El proyecto proporciona información detallada sobre la composición nutricional de los alimentos consumidos comúnmente en Reino Unido.~\cite{cofid2021}

La base de datos ``\textit{Composition of foods integrated dataset (CoFID)}`` se originó a partir del trabajo de Widdowson y McCance en la década de 1940 y ha sido periódicamente actualizada desde entonces para reflejar los últimos cambios en la dieta y las nuevas investigaciones científicas. La última actualización se publicó en marzo de 2021.

La elección de esta base de datos se basa en la gran cantidad de datos nutricionales que se aportan sobre los alimentos, junto con una documentación detallada para entender el trabajo. A todo esto hay que sumar la facilidad para la exportación, que permite modificarla y analizarla con suma sencillez.

CoFID incluye más de 2500 alimentos, cada uno de ellos categorizado según su grupo alimenticio. Cada alimento incorpora información sobre una amplia gama de nutrientes, como vitaminas, minerales, azúcares y otros componentes, además de los macronutrientes (carbohidratos, proteínas y grasas), que van a ser los principalmente usados en este PFG.

En la página web se pueden encontrar principalmente dos archivos. El primero es una guía de usuario en formato PDF que explica en detalle en qué consiste la base de datos y la información que se va a encontrar en ella.\newpage El segundo archivo trata de una hoja de cálculo de Microsoft Excel donde se encuentra toda la información nutricional relativa a 2.887 alimentos. Existen múltiples casos donde un mismo alimento se presenta con diferentes tipos de cocción o de aliño, lo que cambia los valores nutricionales y se considera como dos alimentos o platos distintos.

La información necesaria se encuentra en la tabla ``\textit{1.3. Proximates}``, que es donde se localizan los campos que se van a usar: \textit{''Food Code''}, identificador del alimento; \textit{''Food Name''}, nombre del alimento; \textit{''Group''}, categoría alimenticia a la que pertenece; \textit{''Protein(g)''}, que muestra los gramos de proteína por cada \textit{100g} del alimento; \textit{''Fat(g)''}, lo mismo que la proteína, pero con la grasa; \textit{''Carbohydrate(g)''}, al igual que los dos últimos campos, muestra datos de los hidratos de carbono o carbohidratos; y \textit{''Energy(kcal)''}, que representa las kilocalorías del alimento por cada \textit{100g}.

Como se puede comprobar por las descripciones previas, todos los datos están representados en base a \textit{100} gramos del alimento correspondiente. Esto también ocurre con las bebidas, cuyos valores nutricionales se han calculado en base \textit{100} mililitros. La planificación nutricional se realizará usando estas proporciones.

Por último, destacar que, debido a que la información es extraída de una web oficial del Reino Unido, todos los datos, incluyendo entre ellos el nombre de los alimentos o platos, están en inglés. En el menú de comidas resultante de la ejecución del algoritmo genético se mantendrá este idioma.

\subsection{Creación de la base de datos}
Tras haber seleccionado la fuente, se necesita un gestor de bases de datos para manipular los datos de forma eficiente. Para este trabajo se ha seleccionado MySQL Workbench, que dispone de una interfaz gráfica intuitiva para diseñar, modelar y administrar bases de datos MySQL.~\cite{mysqlworkbench}

Dentro del gestor, lo primero que se debe hacer es crear la base de datos. Se hace uso de sentencias SQL para ello. A la base de datos se le ha llamado \textit{''food\_database''}.

\newpage

Después de configurar la base de datos, se debe crear una tabla con los campos donde se almacenarán los valores. Mostrando en el listado \ref{lst:tabla} el código ejecutado, esta será la única tabla necesaria a lo largo del proyecto.

\begin{lstlisting}[style=sqlstyle, caption=Creación de la tabla y sus campos., label={lst:tabla}]
    CREATE DATABASE food_database;
    CREATE TABLE comida (
        id VARCHAR(10) NOT NULL,
        nombre VARCHAR(255) NOT NULL,
        grupo CHAR(3),
        proteinas FLOAT,
        grasas FLOAT,
        carbohidratos FLOAT,
        calorias INT,
        PRIMARY KEY(id)
    );
\end{lstlisting}

Ahora que la estructura ya está creada, falta introducir los datos. MySQL Workbench dispone de una herramienta para la importación desde archivos Comma Separated Values (valores separados por comas) o CSV, que permite con facilidad identificar los numerosos datos de la hoja de cálculo. Esto permite emparejar fácilmente los campos de la tabla recién creada con los del archivo CSV, como muestra la figura \ref{fig:importacion}, logrando una importación masiva de datos de manera correcta.

\begin{figure}[H]
    \centering
    \includegraphics[width=0.4\textwidth]{figures/importacion.png}
    \caption{Importación de los datos usando MySQL Workbench.}
    \label{fig:importacion}
\end{figure}

\subsection{Cambios en la base de datos original}

Para optimizar la base de datos y mejorar el funcionamiento del algoritmo con el fin de alcanzar mejores soluciones, existen ciertos cambios a realizar a los datos aportados por el gobierno de Reino Unido.

El primer cambio es la eliminación de los datos incompletos. En la base de datos original existen diversos alimentos que no presentan toda la información que se va requerir en el trabajo. El principal promotor de este cambio es la existencia de registros que no tienen calorías asignadas (representadas en el archivo Excel con una \textit{''N''}), que han pasado con el valor \textit{0}.

Sin dejar de lado la eliminación de registros, se filtra las categorías de comidas que pueden ser ingeridas por el consumidor. Existen categorías como, por ejemplo, \textit{''Harinas, granos y almidones''} (AA) o \textit{''Grasas y aceites''} (O), que no son consumidas por el usuario como plato único, sino que participan en la cocción o en el acompañamiento del alimento principal. Por lo tanto, se eliminan. También se elimina la categoría \textit{''Bebidas Alcohólicas''} (Q), ya que sería necesario un análisis más exhaustivo y conocer las recomendaciones de los nutricionistas respecto a su consumo. De esta categoría se mantienen la cerveza, el vino, la sidra y el cava, presentes en la dieta mediterránea y de las que la Sociedad Española de Nutrición Comunitaria (SENC) recomienda el consumo opcional, moderado y responsable~\cite{senpiramide}.

También se ha modificado el grupo \textit{''Zumos''} (PE), donde \textit{PE} se ha cambiado por \textit{FE}. Se ha movido al grupo \textit{''Frutas''} (F) porque los zumos son mucho más parecidos a los alimentos de este grupo que al de las bebidas (P).

En el apéndice \ref{ch:grupos-comida} se encuentran todos los grupos con los que se va a trabajar.

Tras los cambios realizados queda un total de 2616 registros. En la figura \ref{fig:ejemplo} se puede ver un ejemplo de algunos de los alimentos.

\begin{figure}[H]
    \centering
    \includegraphics[width=0.65\textwidth]{figures/ejemplo.png}
    \caption{Resultado final de la base de datos.}
    \label{fig:ejemplo}
\end{figure}

\section{Conceptos nutricionales}

Se ha hecho uso de distintas fuentes para los valores nutricionales utilizados para el cálculo de los objetivos, restricciones y límites que se representan en el problema.

En el algoritmo se busca crear un menú equilibrado y personalizado para el usuario, con el propósito de mantener el peso y la masa corporal del individuo.

\subsection{Cálculo de calorías}
\label{ch:calculo-calorías}

El primer objetivo es el calórico, explicado en el apartado \ref{ch:objetivo-restriccion-calorías}. Para ello es necesario saber cuántas kilocalorías diarias necesita el usuario. La Tasa Metabólica Basal (TMB) es la cantidad de energía que el cuerpo necesita para mantener las funciones vitales en reposo. Existen diversas fórmulas de calcularla, pero en este proyecto se hace uso de una las más extendidas actualmente, la fórmula de Harris-Benedict revisada por Mifflin et al.~\cite{mifflin1990}. Esta fórmula toma en cuenta el peso, la altura, la edad y el sexo para estimar las kilocalorías.

\begin{itemize}
    \item Para hombres:
    \[
    \text{TMB} = (10 \times \text{peso en kg}) + (6.25 \times \text{altura en cm}) - (5 \times \text{edad en años}) + 5
    \]
    \item Para mujeres:
    \[
    \text{TMB} = (10 \times \text{peso en kg}) + (6.25 \times \text{altura en cm}) - (5 \times \text{edad en años}) - 161
    \]
\end{itemize}

Como se explica en la definición de la TMB, esta energía es calculada en reposo. Por lo tanto, para alcanzar una aproximación real será necesario multiplicar el resultado por un factor que depende del nivel de actividad física.~\cite{krause2016}

\begin{itemize}
    \item Sedentario (poco o ningún ejercicio): \textit{TMB $\times$ 1.2}
    \item Actividad ligera (ejercicio ligero o deportes 1-3 días a la semana): \textit{TMB $\times$ 1.375}
    \item Actividad moderada (ejercicio moderado o deportes 3-5 días a la semana): \textit{TMB $\times$ 1.55}
    \item Actividad alta (ejercicio intenso o deportes 6-7 días a la semana): \textit{TMB $\times$ 1.725}
    \item Actividad muy alta (ejercicio muy intenso, trabajos físicos o entrenamiento dos veces al día): \textit{TMB $\times$ 1.9}
\end{itemize}

Por ejemplo, para un hombre de \textit{23 años} que pesa \textit{75 kg} y mide \textit{175 cm} su TMB sería de \textit{1734 kilocalorías por día}. Pero como también realiza una actividad moderada, habría que multiplicar por \textit{1.55}, dando un total de \textit{2687.31 kcal/día}.

\begin{comment}
\subsection{Distribución de calorías}

El segundo objetivo es la correcta distribución diaria de calorías. Se busca que los alimentos ingeridos en cada comida (desayuno, tentempié, almuerzo, merienda y cena) estén dentro de unos límites que sean recomendables. Estos límites son, siguiendo las pautas del Ministerio de Sanidad~\cite{alimentacion_saludable}:

\begin{itemize}
    \item $20\% = \% \text{ de kcal en el desayuno}$
    \item $5\% \leq \% \text{ de kcal en el tentempié} \leq 10\%$
    \item $30\% = \% \text{ de kcal en el almuerzo}$
    \item $5\% \leq \% \text{ de kcal en la merienda} \leq 10\%$
    \item $25\% \leq \% \text{ de kcal en la cena} \leq 30\%$
\end{itemize}

Para calcular el porcentaje de nuestras calorías bastaría con sumar los alimentos correspondientes a una comida en particular, dividirlo entre el número de calorías totales consumidas en ese día y multiplicar el resultado por \textit{100}. La fórmula en una comida como el almuerzo es:
\[
\left( \frac{\text{kcal alimento 1} + \text{kcal alimento 2} + \text{kcal bebida}}{\text{kcal totales}} \right) \times 100
\]
\end{comment}

\subsection{Distribución de macronutrientes}
\label{ch:distribucion-macronutrientes}

Otro objetivo, tratado en el punto \ref{ch:objetivo-restriccion-macronutrientes}, es la correcta distribución diaria de macronutrientes: hidratos de carbono, proteínas y grasas. El \textit{Institute of Medicine (IOM)}, en su trabajo de 2005~\cite{iom2005}, define los AMDR (Acceptable Macronutrient Distribution Ranges) como los rangos de ingesta para macronutrientes que se asocian con un riesgo reducido de enfermedades crónicas y aseguran una ingesta adecuada de nutrientes esenciales. Estos límites para adultos son:

\begin{itemize}
    \item $45\% \leq \% \text{ de carbohidratos} \leq 65\%$
    \item $10\% \leq \% \text{ de proteínas} \leq 35\%$
    \item $20\% \leq \% \text{ de grasas} \leq 35\%$
\end{itemize}

Para verificar si la solución respeta estos límites, primero es necesario calcular cuántas kilocalorías aporta cada macronutriente. Siguiendo el estudio del \textit{IOM}, las kilocalorías por gramo de cada macronutriente son:

\begin{itemize}
    \item Carbohidratos: \textit{4 kcal} por gramo
    \item Proteínas: \textit{4 kcal} por gramo
    \item Grasas: \textit{9 kcal} por gramo
\end{itemize}

La suma de las kilocalorías individuales de cada macronutriente equivale al total de calorías del alimento, dato del que ya se dispone.

Con las kilocalorías de cada macronutriente ya calculadas, solo falta dividir estas entre las calorías totales y multiplicarlo por \textit{100} para saber los porcentajes de cada macronutriente. Por ejemplo, para calcular el porcentaje de carbohidratos sería:
\[
\left( \frac{\text{kcal de carbohidratos}}{\text{kcal totales}} \right) \times 100
\]

Siguiendo el caso de los hidratos de carbono, \textit{11 g} de carbohidratos de un alimento de \textit{98 kcal} representa un porcentaje del \textit{44.9 \%} respecto al total de calorías.


\section{Cromosoma planteado}
\label{ch:solucion-planteada}

Se ha buscado plantear un menú siguiendo los hábitos alimenticios y nutricionales de los consumidores españoles.

La primera parte es la elección de la estructura del cromosoma solución. En este problema se ha planteado que cada gen represente un alimento diferente consumido. Como se muestra en la figura \ref{fig:cromosoma_desarrollo}, cada día del menú semanal incluye 5 comidas: desayuno, tentempié, almuerzo, merienda y cena. De estas comidas, las tres principales, el desayuno, el almuerzo y la cena, consta cada una de 2 alimentos y 1 bebida, haciendo un total de 3 genes. Las otras dos comidas, tentempié y merienda, consideradas como comidas entre horas, contienen 1 alimento cada una, sumando 2 genes adicionales al cromosoma.

Cada día presenta al final 11 genes (o alimentos). Como el proyecto trata de una planificación semanal, se repite esta cadena por cada día de la semana, dando un cromosoma que presenta un total de 77 genes.

\begin{figure}[H]
    \centering
    \includegraphics[width=1\textwidth]{figures/cromosoma_desarrollo.png}
    \caption{Estructura de la solución.}
    \label{fig:cromosoma_desarrollo}
\end{figure}

Como se ha explicado, el tercer gen de cada comida principal está reservado para una bebida, por lo que es necesario filtrar los grupos de comida para que seleccione solo un alimento de esta categoría. Estos cambios que se realizan en la inicialización y mutación son explicados en el apartado \ref{ch:explicacion-algoritmo}, pero primero se define qué categorías de alimentos (detalladas en el apéndice \ref{ch:grupos-comida}) están reservadas para ciertos genes o comidas.

\begin{itemize}
    \item El tercer gen de cada comida principal está destinado a una bebida. De las categorías disponibles, solo se pueden seleccionar de los grupo \textit{''Bebidas''} (P), \textit{''Jugos de frutas''} (FC), \textit{''Zumos''} (FE) o \textit{''Bebidas Alcohólicas''} (Q) (solo si el sujeto es mayor de edad).
    \item Debido a que la bebida del desayuno suele ser una bebida con leche (no presente en el grupo \textit{P}), café o un zumo, se ha decidido que este gen sea de las categorías \textit{''Leche de vaca''} (BA), \textit{''Bebidas a base de leche''} (BH), \textit{''Bebidas en polvo, esencias e infusiones''} (PA), \textit{''Jugos de frutas''} (FC) o \textit{''Zumos''} (FE).
    \item También los alimentos del desayuno suelen diferenciarse con respecto a los del almuerzo y la cena, que son comúnmente intercambiables entre sí. Por ello, los alimentos de esta comida son del grupo \textit{''Huevos''} (C), \textit{''Frutas''} (FA), \textit{''Bacon''} (MAA) o \textit{''Cereales''} (A), excepto \textit{''Arroz''} (AC), \textit{''Pasta''} (AD) y \textit{''Pizza''} (AE), que se suelen consumir en el almuerzo o la cena.
    \item Para las comidas entre horas (tentempié y merienda), se seleccionan alimentos de las categorías \textit{''Frutas''} (F) y \textit{''Azúcares''} (S).
    \item Para las otras dos comidas principales, el almuerzo y la cena, los alimentos pueden ser de cualquier grupo, excepto los cereales, salvo \textit{''Panes''} (AF) y las excepciones antes mencionadas: el arroz, la pasta y las pizzas.
\end{itemize}


\section{Objetivos y restricciones}
\label{ch:objetivo-restricciones}
\begin{comment}
Para los objetivos se crean funciones de minimización, donde la solución candidata (o soluciones) con el valor más bajo es la que mayor aptitud presenta.
\end{comment}
\subsection{Objetivo y restricción de calorías}
\label{ch:objetivo-restriccion-calorías}

El primer objetivo introducido en el problema es de las kilocalorías. Se busca que la diferencia entre las kilocalorías que el usuario necesita diariamente y la suma de las kilocalorías de todos los alimentos consumidos en un día sea 0 ó lo más cercano posible. La función de evaluación mide cuánto se ha desviado en total en toda la semana, es decir, se va sumando cada desviación de kilocalorías diaria hasta obtener un valor semanal.

Se consulta al usuario por distintos atributos físicos y su nivel de actividad, con los cuales se calcula la tasa metabólica basal y, en última instancia, las kilocalorías, como se explica en el apartado \ref{ch:calculo-calorías}.
\newpage
El objetivo está atado a una restricción \textit{box-constraint}. Se penalizan aquellas desviaciones diarias que superen los límites superior e inferior del 10\% respecto a las kilocalorías que el usuario necesita. Busca que las soluciones que se alejen bastante del objetivo sean menos seleccionadas.
\begin{small}
\[
    \begin{aligned}
    & \text{Minimizar } f_{\text{calorías}}(c_{i,d}) = \sum_{d=1}^{7} \left| k - \sum_{i=1}^{n_d} c_{i,d} \right| \\
    & \text{Sujeto a } 0.9k \leq \sum_{i=1}^{n_d} c_{i,d} \leq 1.1k, \quad \forall d \in \{1, 2, \ldots, 7\}
    \end{aligned}
    \]

        Donde:
        \begin{itemize}
        \item \( c_{i,d} \) es la cantidad de calorías del alimento \( i \) consumido en el día \( d \).
        \item \( k \) es el objetivo calórico diario.
        \item \( n_d \) es el número total de alimentos consumidos en el día \( d \).
        \end{itemize}
\end{small}


\subsection{Objetivo y restricción de macronutrientes}
\label{ch:objetivo-restriccion-macronutrientes}

Este objetivo trata de distribuir correctamente los macronutrientes a lo largo del día. Se busca que cada tipo de macronutriente se encuentre dentro de los límites recomendados en el apartado \ref{ch:distribucion-macronutrientes}, poniendo como objetivo la media de estos límites. Similar al anterior objetivo, se calcula la diferencia diaria entre estas medias y los valores reales de proteínas, carbohidratos y grasas obtenidos. Posteriormente, la función evalúa la desviación semanal total.

Existe también una restricción de caja asociada a este objetivo, que exige que la ingesta diaria de cada macronutriente se mantenga dentro de los límites especificados.
\begin{small}
    \[
    \begin{aligned}
    & \text{Minimizar } f_{\text{macronutrientes}} = \sum_{d=1}^{7} \Bigg( \left| \frac{\sum_{i=1}^{n_d} C_{i,d}}{\sum_{i=1}^{n_d} E_{i,d}} - 0.55 \right| \\
    & \qquad + \left| \frac{\sum_{i=1}^{n_d} P_{i,d}}{\sum_{i=1}^{n_d} E_{i,d}} - 0.225 \right| + \left| \frac{\sum_{i=1}^{n_d} G_{i,d}}{\sum_{i=1}^{n_d} E_{i,d}} - 0.275 \right| \Bigg) \\
    & \text{Sujeto a } \\
    & \quad 0.45 \leq \frac{\sum_{i=1}^{n_d} C_{i,d}}{\sum_{i=1}^{n_d} E_{i,d}} \leq 0.65, \\
    & \quad 0.10 \leq \frac{\sum_{i=1}^{n_d} P_{i,d}}{\sum_{i=1}^{n_d} E_{i,d}} \leq 0.35, \\
    & \quad 0.20 \leq \frac{\sum_{i=1}^{n_d} G_{i,d}}{\sum_{i=1}^{n_d} E_{i,d}} \leq 0.35, \\
    & \quad \forall d \in \{1, 2, \ldots, 7\}
    \end{aligned}
    \]
\newpage
        Donde:
        \begin{itemize}
        \item \( E_{i,d} \) es la cantidad de energía (calorías) del alimento \( i \) consumido en el día \( d \).
        \item \( C_{i,d} \) es la cantidad de calorías provenientes de carbohidratos del alimento \( i \) consumido en el día \( d \).
        \item \( P_{i,d} \) es la cantidad de calorías provenientes de proteínas del alimento \( i \) consumido en el día \( d \).
        \item \( G_{i,d} \) es la cantidad de calorías provenientes de grasas del alimento \( i \) consumido en el día \( d \).
        \item \( n_d \) es el número total de alimentos consumidos en el día \( d \).
        \end{itemize}
\end{small}

\subsection{Objetivo de preferencia}

Este objetivo busca la personalización del menú incluyendo los grupos de alimentos favoritos del usuario, o excluyendo aquellos que no le gustan. Se favorecerán los grupos preferidos y se penalizarán los que no. Se puede considerar una restricción de igualdad débil, ya que penaliza las soluciones que la violan pero no las invalida. A diferencia de los otros objetivos, esta función de minimización puede alcanzar valores negativos si existe una gran variedad de alimentos con grupo predilecto en la planificación.
\[
\text{Minimizar } f_{\text{preferencia}} = \sum_{i=1}^{N}
\begin{cases} 
-P & \text{si } g(a_i) = G_+ \\
P & \text{si } g(a_i) = G_- \\
0 & \text{en otro caso}
\end{cases}
\]
\begin{small}
    Donde:
    \begin{itemize}
    \item \( f_{\text{preferencia}} \) es la penalización total de preferencia de grupo durante la semana.
    \item \( N \) es el número total de alimentos consumidos en la semana.
    \item \( a_i \) es el alimento \( i \) consumido durante la semana.
    \item \( G_+ \) es el grupo que gusta.
    \item \( G_- \) es el grupo que no gusta.
    \item \( P \) es la penalización de preferencia.
    \item \( g(a_i) \) es el grupo del alimento \( a_i \).
    \end{itemize}
\end{small}

\subsection{Restricción de alergia}

Similar al objetivo de preferencia, es una restricción de igualdad donde se penaliza severamente si el usuario es alérgico al grupo del alimento del menú. Al finalizar la semana, se suman todas las penalizaciones de alergias, haciendo que sea inviable para el algoritmo volver a seleccionar un alimento que produzca alergia.
\[
\text{Minimizar } f_{\text{alergia}} = \sum_{i=1}^{N} 
\begin{cases} 
P_{\text{alergia}}^2 & \text{si } g(a_i) = G_{\text{alergia}} \\
0 & \text{en otro caso}
\end{cases}
\]
\begin{small}
    Donde:
    \begin{itemize}
    \item \( f_{\text{alergia}} \) es la penalización total por alergia durante la semana.
    \item \( N \) es el número total de alimentos consumidos en la semana.
    \item \( a_i \) es el alimento \( i \) consumido durante la semana.
    \item \( G_{\text{alergia}} \) es el grupo al que el usuario es alérgico.
    \item \( P_{\text{alergia}} \) es la penalización fuerte por alergia.
    \item \( g(a_i) \) es el grupo del alimento \( a_i \).
    \end{itemize}
\end{small}

\section{Construcción del algoritmo}
\label{ch:explicacion-algoritmo}

El código completo se encuentra en el repositorio de Github del autor~\cite{quesada_nutritionplanning}.

La biblioteca usada para la creación del algoritmo evolutivo ha sido \textit{Pymoo}. \textit{Pymoo} es un framework de \textit{Python} diseñado para la optimización multiobjetivo. Entre sus ventajas se encuentran la amplia gama de algoritmos para abordar los problemas, como \textit{NSGA-II}, \textit{SPEA2} o \textit{MOEA/D}. Permite modificar fácilmente los operadores de selección, cruce y mutación, lo que ayuda a la personalización del algoritmo según el problema. Además, \textit{Pymoo} presenta un conjunto de herramientas de visualización para la interpretación de resultados, permitiendo entender el comportamiento de los algoritmos.~\cite{pymoo}

Se ha creado una clase personalizada que hereda de \textit{''ElementwiseProblem''}, ya que esta última proporciona una estructura eficiente para evaluar cada solución individualmente, como se muestra en el listado \ref{lst:planning}.
\newpage
\begin{lstlisting}[basicstyle=\ttfamily, caption=Clase para la evaluación.,label={lst:planning}]
    CLASE PlanningComida hereda ElementwiseProblem
    
        MÉTODO __init__(self, num_variables, limites_variables,
                            num_objetivos, num_restricciones)
            INICIALIZAR num_variables
            INICIALIZAR limites_variables
            INICIALIZAR num_objetivos
            INICIALIZAR num_restricciones
            LLAMAR al constructor de ElementwiseProblem
    
        MÉTODO evaluate(self, solución, out)
            CALCULAR valores_objetivo
            CALCULAR valores_restricciones
            out["F"] = valores_objetivo
            out["G"] = valores_restricciones
\end{lstlisting}

El primer método creado es \textit{''\_\_init\_\_''}, que inicializa los parámetros del problema, incluyendo el tamaño de la solución (77 genes), el número de objetivos y restricciones (3 cada uno) y los límites del problema (el tamaño de la base de datos).

El segundo método es \textit{''evaluate''}, que es donde se define cómo se evalúan las soluciones. Este método recibe una solución individual y calcula su aptitud a partir de los objetivos y restricciones definidos en el apartado \ref{ch:objetivo-restricciones}.

Se recorre la cadena y, por cada gen, se extraen las calorías, el grupo y los macronutrientes. Se comprueba el objetivo de la preferencia comparando el grupo del alimento con los grupos de preferencia del usuario y, en caso de ser necesario, se suma una penalización a una variable que contará todas las sanciones totales de la solución respecto a este objetivo. El mismo procedimiento se hace con la restricción de la alergia.

Por cada 11 genes de la cadena (que representa un día) se comprueba el objetivo calórico, donde la desviación de kilocalorías diarias respecto al objetivo marcado para el usuario se suma a un contador que refleja la desviación semanal total. Este es el valor a minimizar. Pero, además, existe otro contador que suma todas las penalizaciones de la restricción calórica. Cada vez que las kilocalorías diarias se sale de los límites del 10\% se agrega una penalización a este contador. Este es otro valor que se busca minimizar.

El cálculo de los valores a minimizar del objetivo y restricción de macronutrientes sigue el mismo proceso que el de las calorías.

Los resultados de las evaluaciones se almacenan en los diccionarios \textit{''out[F]''} y \textit{''out[G]''}. \textit{''out[F]''} contiene los valores de las funciones objetivos, que intentamos optimizar. \textit{''out[G]''}, muy usado en métodos separatistas para el manejo de restricciones, contiene los valores de las restricciones que deben ser satisfechas para que las soluciones se consideren factibles. Si se usase, por ejemplo, métodos de penalización para las restricciones, no sería necesario incluir \textit{''out[G]''}, ya que las restricciones se tratan como funciones objetivo. 

Tras la creación de la clase, hay que crear una función que recoja los resultados de esta clase para incluirlos en el algoritmo evolutivo. El funcionamiento del algoritmo es, en esencia, el mismo que se ha explicado en la sección \ref{ch:marco-teorico}, aunque se puede modificar o añadir algún operador según el tipo de algoritmo usado, razón de la experimentación con distintos algoritmos del apartado \ref{ch:algoritmos-multiobjetivo}. Por lo tanto, lo primero a seleccionar es el algoritmo con el que se va a trabajar, ya sea NSGA-II, MOEA/D, etc.

Se selecciona el número de individuos por generación y el número de generaciones que el algoritmo evaluará. Como no se ha puesto ninguna condición de parada específica el algoritmo terminará tras haber evaluado todas las generaciones predefinidas.

Lo primero que el algoritmo hace es crear la matriz en la que se aloja toda la población, que tiene un tamaño del número de genes (o variables) por solución multiplicado por el número de individuos de una generación. Por ejemplo, si el menú se compone de \textit{77} genes y se selecciona un tamaño de población de \textit{100} indiviudos, se crea una matriz de \(100 \, individuos \times 77 \, genes\), en la que cada solución candidata se evalúa individualmente.

\textit{Pymoo} dispone de varios métodos de incialización, entre los que destaca \textit{''IntegerRandomSampling''}, usado para la generación de matrices de enteros. Este se puede usar en el planificador de comidas al seleccionar el índice del alimento que se añade al menú, el cual es un número entero. Pero se ha decidido crear una versión modificada del mismo para que genere soluciones que sean interesantes en el contexto del proyecto. En vez de seleccionar un índice aleatorio de un alimento de la base de datos, selecciona basándose en las características propias de la solución, explicadas en el apartado \ref{ch:solucion-planteada}. Es decir, se ha configurado para que se seleccione una bebida cuando el gen sea el tercero de una comida principal o para que los alimentos del desayuno sean de la categoría \textit{''Cereales''} (A), por ejemplo. Así se consigue que la primera población de la generación cumpla las particularidades del problema, como se observa en la figura \ref{fig:matriz-inicializacion}.

\begin{figure}[H]
    \centering
    \includegraphics[width=0.75\textwidth]{figures/matriz-inicializacion.png}
    \caption{Matriz de inicialización.}
    \label{fig:matriz-inicializacion}
\end{figure}

Cada solución de esta población es evaluada según lo explicado al comienzo de este apartado. Si no ha llegado al número total de generaciones, el algoritmo se prepara para generar una nueva población a partir de los operadores.

En la selección se eligen individuos de la población basado en la aptitud. En este caso concreto los valores más bajos presentarán una mayor aptitud, al ser funciones de minimización. Se va a usar la selección por torneo, predeterminada por \textit{Pymoo} para este tipo de problemas.

Tras elegir los individuos, se cruzan para la generación de una nueva población.  Se puede variar entre el cruce de un punto y el de dos puntos, asignándoles a cada uno de ellos una probabilidad de cruce \(P_c\)  . Aparte de los mecanismos elitistas que usen los algoritmos multi-objetivo, no se ha realizado ningún método extra de selección ambiental, por lo que se realiza un reemplazo generacional completo.

Para la mutación ocurre como con la inicialización. En vez de elegir un método de los que ofrece \textit{Pymoo}, se ha decidido crear un método personalizado. El nuevo método recorre cada individuo creado en el cruce y decide si el gen es reservado para una bebida, para un snack, etc. Con una probabilidad \(P_m\) a determinar, muta una bebida por otra bebida o un alimento de una categoría determinada por otro de la misma categoría, como se muestra en la figura \ref{fig:mutacion-custom}. Esto logra, junto con la inicialización, que todas las soluciones a lo largo de las generaciones cumplan con las peculiaridades de este problema.

\begin{figure}[H]
    \centering
    \includegraphics[width=1\textwidth]{figures/mutacion-custom.png}
    \caption{Mutación personalizada.}
    \label{fig:mutacion-custom}
\end{figure}

Por último, esta nueva población resultante vuelve a ser evaluada en busca de ser óptima para el problema.

El algoritmo evolutivo se ejecuta utilizando la función \textit{''minimize''} de \textit{Pymoo}, que toma el problema y la configuración del algoritmo. Como se muestra en la tabla \ref{tab:optimizacion}, el algoritmo va avanzando en busca de mejores soluciones.

En la tabla se aprecia, en orden: \textit{''n\_gen''} muestra el número de generación, \textit{''n\_eval''} es el contador de evaluaciones de individuos, \textit{''n\_nds''} es el número de soluciones no dominadas encontradas, \textit{''cv\_min''} es el valor mínimo de violación de una población, \textit{''n\_avg''} es el valor promedio de violación de una población, \textit{''eps''} es una métrica que mide la convergencia del algoritmo hacia soluciones óptimas e \textit{''indicator''} se usa para clasificar la calidad del frente de Pareto en cada generación.

\begin{table}[h!]
    \centering
    \small % Reduce el tamaño del texto
    \begin{tabularx}{\textwidth}{@{}ccccccc@{}}
        \toprule
        \textbf{n\_gen} & \textbf{n\_eval} & \textbf{n\_nds} & \textbf{cv\_min} & \textbf{cv\_avg} & \textbf{eps} & \textbf{indicator} \\ 
        \midrule
         1 &   100 &  1 & 2.415619E+03 & 5.717019E+03 &          - &          - \\ 
         2 &   200 &  1 & 1.789955E+03 & 4.574975E+03 &          - &          - \\ 
         3 &   300 &  1 & 1.290943E+03 & 3.787494E+03 &          - &          - \\ 
         4 &   400 &  1 & 1.195388E+03 & 3.107680E+03 &          - &          - \\ 
         5 &   500 &  1 & 1.070258E+03 & 2.478680E+03 &          - &          - \\ 
         6 &   600 &  1 & 0.000000E+00 & 1.867363E+03 & 1.0000000000 &     ideal \\ 
         7 &   700 &  2 & 0.000000E+00 & 1.400831E+03 & 0.5000000000 &     ideal \\ 
         8 &   800 &  3 & 0.000000E+00 & 9.870077E+02 & 0.2500000000 &         f \\ 
         9 &   900 &  4 & 0.000000E+00 & 6.756844E+02 & 0.1250000000 &     ideal \\ 
        10 &  1000 &  5 & 0.000000E+00 & 4.510310E+02 & 0.0625000000 &     ideal \\ 
        \bottomrule
    \end{tabularx}
    \caption{Resultados del algoritmo genético.}
    \label{tab:optimizacion}
\end{table}

El algoritmo, con este manejo de restricciones, está configurado para que, hasta que no alcance una solución viable (\textit{''cv\_min''}=0), no empieza a evaluar la convergencia y la calidad del frente de Pareto.

En el caso de que, como se ha explicado antes, no fuera necesario \textit{''out[G]''} por el manejo de las restricciones, simplemente desaparecen \textit{''cv\_min''} y \textit{''cv\_avg''}, y \textit{''n\_nds''}, \textit{''eps''} e \textit{''indicator''} se calculan desde la primera generación.

Todas las métricas que obtiene el algoritmo son guardadas en una variable \textit{''resultado''}, a la que se puede acceder.

\begin{itemize}
    \item En \textit{''resultado.X''} se encuentra una matriz que contiene las soluciones no dominadas encontradas. En este proyecto son cada uno de los posibles menús factibles que el algoritmo ha encontrado. Siguiendo el ejemplo \ref{tab:optimizacion}, en 10 generaciones el algoritmo ha encontrado 5 soluciones no dominadas, por lo que la matriz tiene un tamaño de \(5 \, n\_nds\ \times 77 \, genes\).
    \item En \textit{''resultado.F''} se encuentra una matriz que contiene los valores de las funciones objetivo para cada solución no dominada encontrada. Como el algoritmo presenta 3 objetivos, la matriz puede tener un tamaño de \(5 \, n\_nds \times 3 \, objetivos\).
    \item En \textit{''resultado.G''} se encuentra una matriz muy parecida a \textit{''resultado.F''}, pero con las restricciones en vez de los objetivos. Esta matriz se encuentra vacía si los objetivos y las restricciones no se tratan de manera separada.
\end{itemize}

\subsection{Manejo de restricciones}
\label{ch:manejo-restricciones}

\subsubsection{Penalización estática}
\label{ch:penalizacion-estatica}

En este método se penalizan las soluciones que violan las restricciones. En las funciones en las que se definen las restricciones de calorías y macronutrientes, se calcula una penalización proporcional a la diferencia entre el objetivo y los nutrientes consumidos, multiplicada por una constante de penalización, como se muestra en el listado \ref{lst:factores}. Poniendo de ejemplo la restricción de calorías, si el objetivo calórico es 2500 kilocalorías y la diferencia con las kilocalorías ingeridas está fuera del 10\%, se multiplica esa diferencia por un factor de penalización. Por lo tanto, cuanto mayor sea la diferencia, mayor será la penalización, lo que guía al algoritmo hacia soluciones con menor penalización y más cercanas al objetivo.

En el caso de la restricción de alergia, se aplica un factor de penalización si el grupo alérgico aparece en el menú. Si no aparece, se devuelve 0. Tras realizar diversos tests, se ha llegado a la conclusión de que el algoritmo encuentra mejores resultados si el factor de penalización se eleva al cuadrado, lo que hace inviable que el algoritmo seleccione soluciones que incumplan esta restricción.
\newpage
En el objetivo de preferencias, que se puede considerar una restricción débil, se penaliza con un factor de penalización (de menor valor que el de la alergia) en el caso de que aparezca en el menú un grupo de alimentos que disguste al usuario. En cambio, si el alimento es del gusto del individuo, se devuelve ese mismo factor con un signo negativo delante, lo que ayuda a que baje el valor a minimizar.

Los valores retornados de las funciones de restricción se suman entre sí, y el resultado se añade a cada uno de los valores objetivo del problema. El algoritmo pasa de calcular 3 objetivos y 3 restricciones a calcular solo los 3 objetivos a los que se le ha sumado la penalización de las restricciones.

\begin{lstlisting}[basicstyle=\ttfamily, caption=Factores de penalización.,label={lst:factores}]
    PENALIZACION_calorías = 50
    PENALIZACION_MACRONUTRIENTES = 30
    PENALIZACION_PREFERENCIA = 10
    PENALIZACION_ALERGIA = 100
\end{lstlisting}

\subsubsection{Método separatista}
\label{ch:metodo-separatista}

En el método separatista, los objetivos y las restricciones se tratan por separado. A diferencia de los métodos de penalización, en las funciones de restricción no se hace uso de los factores. Para las restricciones de calorías y macronutrientes, se devuelve únicamente la diferencia entre el valor objetivo y el valor obtenido de la solución candidata.

La función de la restricción de alergia (y la de preferencias) se maneja igual que en los métodos con penalización, por lo que aquí sí que se usa el factor de penalización correspondiente.

En \textit{''out[F]''} se almacenan los valores de las funciones objetivos (3 en total) y en \textit{''out[G]''} se guardan los valores de las funciones de restriccion (también 3). El algoritmo busca primero soluciones que no violen las restricciones y, cuando las encuentra, busca soluciones que optimicen el problema. En la tabla \ref{tab:optimizacion} se puede ver este funcionamiento.

\subsubsection{Restricciones como objetivos}
\label{ch:restricciones-objetivo}

Si bien en los anteriores casos se desarrollaron versiones personalizadas de los métodos, en este tercer caso se va a hacer uso de la herramienta de \textit{Pymoo} \textit{''ConstraintsAsObjective''}~\cite{pymoo_constraints_as_obj}.

Este método se caracteriza por incorporar las restricciones como objetivos. La finalidad no es solo encontrar soluciones que cumplan con todas las restricciones, sino también evaluar cuánto se puede mejorar la optimización si se relajan las restricciones. La construcción del método de evaluación se realiza igual que en el método separatista. El cambio se lleva a cabo en la función de optimización, donde se indica que en el algoritmo a minimizar se empleará \textit{''ConstraintsAsObjective''}.
\section{Interfaz gráfica}
\label{ch:interfaz-grafica}

Si bien el objetivo principal de este PFG es la construcción del algoritmo evolutivo y la experimentación del mismo (apartado \ref{ch:objetivos}), se ha creado una interfaz sencilla que sirve como guía para la ejecución del algoritmo. Para ello, se ha utilizado la biblioteca \textit{''tkinter''}~\cite{python_tkinter}.

La ventana principal, que es la que se muestra en la figura \ref{fig:ventana-main}, es el punto de entrada para la ejecución del algoritmo. Se ha nombrado \textit{''Planificación nutricional mediante algoritmos evolutivos''}. Dispone de un título, que comparte con el nombre de la ventana, y dos botones. El botón de la izquierda, \textit{''Planificar el menú''}, lleva a la ventana que se encuentra en la figura \ref{fig:ventana-usuario}, donde se consulta al usuario para calcular sus kilocalorías necesarias y sus preferencias alimenticias. El segundo botón, \textit{''Visualizar la base de datos''}, redirige a la ventana \textit{''Base de datos''}, mostrada en la figura \ref{fig:ventana-basedatos}. Aquí se indica cada alimento con su grupo y sus valores nutricionales.

\begin{figure}[H]
    \centering
    \includegraphics[width=0.9\textwidth]{figures/ventana-main.png}
    \caption{Ventana principal.}
    \label{fig:ventana-main}
\end{figure}

En la ventana \textit{''Base de datos''}, figura \ref{fig:ventana-basedatos}, se ha creado un \textit{''TreeView''} para la representación de los datos. Es una herramienta que permite visualizar y gestionar datos en una estructura de árbol. En este caso tiene una forma de tabla y muestra una lista completa de alimentos, cada uno con su categoría, sus calorías y sus macronutrientes. Además, asociado a este \textit{''TreeView''}, se ha creado un \textit{scrollbar} vertical para poder navegar entre los numerosos alimentos.

Se ha incluido un botón \textit{''Volver''} que redirige a la ventana principal \textit{''Planificación nutricional mediante algoritmos evolutivos''}.

\begin{figure}[H]
    \centering
    \includegraphics[width=0.9\textwidth]{figures/ventana-basedatos.png}
    \caption{Ventana de la base de datos.}
    \label{fig:ventana-basedatos}
\end{figure}

La ventana \textit{''Preguntas al usuario''}, figura \ref{fig:ventana-usuario}, es usada para calcular las kilocalorías que el usuario necesita y sus grupos alimenticios de preferencia.

Se pregunta al usuario por medio de un \textit{''label''} y este responde en un \textit{''entry''}, que almacena la respuesta. Así ocurre con las preguntas sobre el peso, la edad y la altura. Para el sexo y el nivel de actividad se ha hecho uso de un \textit{''combobox''}, que permite introducir varias opciones posibles dentro de una lista. Para la alergia y las preferencias alimenticias se ha utilizado \textit{''listbox''}, que permite mostrar, correctamente tabulados, todos los grupos de alimentos. Se puede hacer selección múltiple. Estos grupos quedan guardados para luego usarlos durante la ejecución del algoritmo.

El botón \textit{''Calcular calorías''} activa el cálculo de las kilocalorías objetivo del usuario y las muestra. El botón \textit{''Mostrar menú''} activa la ejecución del algoritmo para que encuentre una solución óptima. El último botón, \textit{''Volver''}, regresa a la ventana principal.

\begin{figure}[H]
    \centering
    \includegraphics[width=1\textwidth]{figures/ventana-preguntasusuario.png}
    \caption{Ventana de las preguntas al usuario.}
    \label{fig:ventana-usuario}
\end{figure}
\newpage
Tras pulsar el botón \textit{''Mostrar menú''}, el algoritmo se ejecuta con los datos introducidos. Cuando termina, se selecciona la primera de las soluciones no dominadas y se obtiene un menú compuesto por 77 alimentos.

Se ha creado la ventana \textit{''Menú Generado''}, mostrada en la figura \ref{fig:ventana-menu}, donde se ha hecho uso de un \textit{''grid''}, muy útil para la creación de tablas estáticas. Se han introducido 5 filas (comidas diarias) por 7 columnas (días de la semana). Se muestran todos los alimentos que componen la solución elegida, además del grupo al que pertenecen. Se han añadido una fila y una columna como cabeceras, correspondientes a las comidas del día y a los días de la semana, respectivamente. Además, se ha incorporado una fila al final para ayudar a entender si la solución escogida cumple con los objetivos de las kilocalorías y los macronutrientes.

Se han añadido dos botones, \textit{''Atrás''} y \textit{''Cerrar''}. El primero vuelve a la ventana \textit{''Preguntas al usuario''} para la posibilidad de generar otro menú distinto. El segundo botón termina con la ejecución del programa.

\begin{figure}[H]
    \centering
    \includegraphics[width=0.965\textwidth]{figures/ventana-menu.png}
    \caption{Ventana del menú generado.}
    \label{fig:ventana-menu}
\end{figure}
\chapter{Experimentación}
\label{ch:experimentacion}

\section{Manejo de restricciones}
\label{ch:manejo-restricciones}

\subsection{Penalización estática}
\label{ch:penalizacion-estatica}

\subsection{Penalización dinámica}
\label{ch:penalizacion-dinamica}

\subsection{Método separatista}
\label{ch:metodo-separatista}


\section{Variación de algoritmo}
\label{ch:distinto-algoritmo}

\subsection{Non-dominated Sorting Genetic Algorithm II (NSGA-II)}
\label{ch:nsga2}

\subsection{Strength Pareto Evolutionary Algorithm 2 (SPEA2)}
\label{ch:spea2}

\subsection{Multi-Objective Evolutionary Algorithm based on Decomposition (MOEA/D)}
\label{ch:moead}


\appendix

\chapter{Apéndice: Grupos de comida}
\label{ch:grupos-comida}

\begin{small}
    \begin{tabbing}
    \hspace{15cm} \= \hspace{0cm} \kill
        \textbf{Cereales y productos de cereales (Cereals and cereal products)} \> \textbf{A} \\
            \hspace{0.5cm}Harinas, granos y almidones (Flours, grains and starches) \> AA \\
            \hspace{0.5cm}Sándwiches (Sandwiches) \> AB \\
            \hspace{0.5cm}Arroz (Rice) \> AC \\
            \hspace{0.5cm}Pasta (Pasta) \> AD \\
            \hspace{0.5cm}Pizzas (Pizzas) \> AE \\
            \hspace{0.5cm}Panes (Breads) \> AF \\
            \hspace{0.5cm}Panecillos (Rolls) \> AG \\
            \hspace{0.5cm}Cereales de desayuno (Breakfast cereals) \> AI \\
            \hspace{0.5cm}Alimentos infantiles de cereales (Infant cereal foods) \> AK \\
            \hspace{0.5cm}Galletas (Biscuits) \> AM \\
            \hspace{0.5cm}Pasteles (Cakes) \> AN \\
            \hspace{0.5cm}Pastelería (Pastry) \> AO \\
            \hspace{0.5cm}Bollos y pasteles (Buns and pastries) \> AP \\
            \hspace{0.5cm}Pudines (Puddings) \> AS \\
            \hspace{0.5cm}Aperitivos (Savouries) \> AT \\
    \end{tabbing}

    \vspace{-1.25cm}

    \begin{tabbing}
    \hspace{15cm} \= \hspace{0cm} \kill
        \textbf{Leche y productos lácteos (Milk and milk products)} \> \textbf{B} \\
            \hspace{0.5cm}Leche de vaca (Cow's milk) \> BA \\
                \hspace{1cm}Leche para desayuno (Breakfast milk) \> BAB \\
                \hspace{1cm}Leche desnatada (Skimmed milk) \> BAE \\
                \hspace{1cm}Leche semi-desnatada (Semi-skimmed milk) \> BAH \\
                \hspace{1cm}Leche entera (Whole milk) \> BAK \\
                \hspace{1cm}Leche de Channel Island (Channel Island milk) \> BAN \\
                \hspace{1cm}Leches procesadas (Processed milks) \> BAR \\
            \hspace{0.5cm}Otras leches (Other milks) \> BC \\
            \hspace{0.5cm}Fórmulas infantiles (Infant formulas) \> BF \\
                \hspace{1cm}Leches modificadas a base de suero (Whey-based modified milks) \> BFD \\
                \hspace{1cm}Leches modificadas no a base de suero \\
                \hspace{1cm}(Non-whey-based modified milks) \> BFG \\
                \hspace{1cm}Leches modificadas a base de soja (Soya-based modified milks) \> BFJ \\
                \hspace{1cm}Fórmulas de seguimiento (Follow-on formulas) \> BFP \\
            \hspace{0.5cm}Bebidas a base de leche (Milk-based drinks) \> BH \\
            \hspace{0.5cm}Cremas (Creams) \> BJ \\
                \hspace{1cm}Cremas frescas (pasteurizadas) (Fresh creams (pasteurised)) \> BJC \\
                \hspace{1cm}Cremas congeladas (pasteurizadas) (Frozen creams (pasteurised)) \> BJF \\
                \hspace{1cm}Cremas esterilizadas (Sterilised creams) \> BJL \\
                \hspace{1cm}Cremas UHT (UHT creams) \> BJP \\
                \hspace{1cm}Cremas imitadas (Imitation creams) \> BJS \\
            \hspace{0.5cm}Quesos (Cheeses) \> BL \\
            \hspace{0.5cm}Yogures (Yogurts) \> BN \\
                \hspace{1cm}Yogures de leche entera (Whole milk yogurts) \> BNE \\
                \hspace{1cm}Yogures bajos en grasa (Low fat yogurts) \> BNH \\
                \hspace{1cm}Otros yogures (Other yogurts) \> BNS \\
            \hspace{0.5cm}Helados (Ice creams) \> BP \\
            \hspace{0.5cm}Pudines y postres refrigerados (Puddings and chilled desserts) \> BR \\
            \hspace{0.5cm}Platos salados y salsas (Savoury dishes and sauces) \> BV \\
    \end{tabbing}

    \vspace{-1.25cm}
   
    \begin{tabbing}
    \hspace{15cm} \= \hspace{0cm} \kill
        \textbf{Huevos (Eggs)} \> \textbf{C} \\
            \hspace{0.5cm}Huevos (Eggs) \> CA \\
            \hspace{0.5cm}Platos de huevo (Egg dishes) \> CD \\
                \hspace{1cm}Platos de huevo salados (Savoury egg dishes) \> CDE \\
                \hspace{1cm}Platos de huevo dulces (Sweet egg dishes) \> CDH \\
    \end{tabbing}
    
    \vspace{-1.25cm}
    
    \begin{tabbing}
    \hspace{15cm} \= \hspace{0cm} \kill
        \textbf{Verduras (Vegetables)} \> \textbf{D} \\
            \hspace{0.5cm}Patatas (Potatoes) \> DA \\
                \hspace{1cm}Patatas tempranas (Early potatoes) \> DAE \\
                \hspace{1cm}Patatas de cosecha principal (Main crop potatoes) \> DAM \\
                \hspace{1cm}Papas fritas de patatas viejas (Chipped old potatoes) \> DAP \\
                \hspace{1cm}Productos de patata (Potato products) \> DAR \\
            \hspace{0.5cm}Judías y lentejas (Beans and lentils) \> DB \\
            \hspace{0.5cm}Guisantes (Peas) \> DF \\
            \hspace{0.5cm}Verduras, en general (Vegetables, general) \> DG \\
            \hspace{0.5cm}Verduras, secas (Vegetables, dried) \> DI \\
            \hspace{0.5cm}Platos de verduras (Vegetable dishes) \> DR \\
    \end{tabbing}
    
    \vspace{-1.25cm}

    \begin{tabbing}
    \hspace{15cm} \= \hspace{0cm} \kill
        \textbf{Fruta (Fruit)} \> \textbf{F} \\
            \hspace{0.5cm}Fruta, en general (Fruit, general) \> FA \\
            \hspace{0.5cm}Zumos de fruta (Fruit juices) \> FC \\
    \end{tabbing}

    \vspace{-1.25cm}
    
    \begin{tabbing}
    \hspace{15cm} \= \hspace{0cm} \kill
        \textbf{Frutos secos y semillas (Nuts and seeds)} \> \textbf{G} \\
            \hspace{0.5cm}Frutos secos y semillas, en general (Nuts and seeds, general) \> GA \\
    \end{tabbing}
    
    \vspace{-1.25cm}

    \begin{tabbing}
    \hspace{15cm} \= \hspace{0cm} \kill
        \textbf{Pescado y productos de pescado (Fish and fish products)} \> \textbf{J} \\
            \hspace{0.5cm}Pescado blanco (White fish) \> JA \\
            \hspace{0.5cm}Pescado graso (Fatty fish) \> JC \\
            \hspace{0.5cm}Crustáceos (Crustacea) \> JK \\
            \hspace{0.5cm}Moluscos (Molluscs) \> JM \\
            \hspace{0.5cm}Productos y platos de pescado (Fish products and dishes) \> JR \\
    \end{tabbing}

    \vspace{-1.25cm}
    
    \begin{tabbing}
    \hspace{15cm} \= \hspace{0cm} \kill
        \textbf{Carne y productos cárnicos (Meat and meat products)} \> \textbf{M} \\
            \hspace{0.5cm}Carne (Meat) \> MA \\
                \hspace{1cm}Tocino (Bacon) \> MAA \\
                \hspace{1cm}Carne de res (Beef) \> MAC \\
                \hspace{1cm}Cordero (Lamb) \> MAE \\
                \hspace{1cm}Cerdo (Pork) \> MAG \\
                \hspace{1cm}Ternera (Veal) \> MAI \\
            \hspace{0.5cm}Aves (Poultry) \> MC \\
                \hspace{1cm}Pollo (Chicken) \> MCA \\
                \hspace{1cm}Pato (Duck) \> MCC \\
                \hspace{1cm}Ganso (Goose) \> MCE \\
                \hspace{1cm}Urogallo (Grouse) \> MCG \\
                \hspace{1cm}Perdiz (Partridge) \> MCI \\
                \hspace{1cm}Faisán (Pheasant) \> MCK \\
                \hspace{1cm}Paloma (Pigeon) \> MCM \\
                \hspace{1cm}Pavo (Turkey) \> MCO \\
            \hspace{0.5cm}Caza (Game) \> ME \\
                \hspace{1cm}Liebre (Hare) \> MEA \\
                \hspace{1cm}Conejo (Rabbit) \> MEC \\
                \hspace{1cm}Venado (Venison) \> MEE \\
            \hspace{0.5cm}Despojos (Offal) \> MG \\
                \hspace{1cm}Hamburguesas y filetes para parrilla (Burgers and grillsteaks) \> MBG \\
            \hspace{0.5cm}Productos cárnicos (Meat products) \> MI \\
                \hspace{1cm}Otros productos cárnicos (Other meat products) \> MIG \\
            \hspace{0.5cm}Platos de carne (Meat dishes) \> MR \\
    \end{tabbing}

    \vspace{-1.25cm}
    
    \begin{tabbing}
    \hspace{15cm} \= \hspace{0cm} \kill
        \textbf{Bebidas (Beverages)} \> \textbf{P} \\
            \hspace{0.5cm}Bebidas en polvo, esencias e infusiones  \\
            \hspace{0.5cm}(Powdered drinks, essences and infusions) \> PA \\
                \hspace{1cm}Bebidas en polvo y esencias (Powdered drinks and essences) \> PAA \\
                \hspace{1cm}Infusiones (Infusions) \> PAC \\
            \hspace{0.5cm}Bebidas sin alcohol (Soft drinks) \> PC \\
                \hspace{1cm}Bebidas carbonatadas (Carbonated drinks) \> PCA \\
                \hspace{1cm}Concentrados y cordiales (Squash and cordials) \> PCC \\
            \hspace{0.5cm}Zumos (Juices) \> PE \\
    \end{tabbing}

    \vspace{-1.25cm}
    
    \begin{tabbing}
    \hspace{15cm} \= \hspace{0cm} \kill
        \textbf{Bebidas alcohólicas (Alcoholic beverages)} \> \textbf{Q} \\
            \hspace{0.5cm}Cervezas (Beers) \> QA \\
            \hspace{0.5cm}Sidras (Ciders) \> QC \\
            \hspace{0.5cm}Vinos (Wines) \> QE \\
    \end{tabbing}

    \vspace{-1.25cm}
    
    \begin{tabbing}
    \hspace{15cm} \= \hspace{0cm} \kill
        \textbf{Azúcares, conservas y snacks (Sugars, preserves and snacks)} \> \textbf{S} \\
            \hspace{0.5cm}Azúcares, jarabes y conservas (Sugars, syrups and preserves) \> SC \\
            \hspace{0.5cm}Confitería (Confectionery) \> SE \\
                \hspace{1cm}Confitería de chocolate (Chocolate confectionery) \> SEA \\
                \hspace{1cm}Confitería sin chocolate (Non-chocolate confectionery) \> SEC \\
            \hspace{0.5cm}Snacks salados (Savoury snacks) \> SN \\
                \hspace{1cm}Snacks a base de patata (Potato-based snacks) \> SNA \\
                \hspace{1cm}Snacks de patata y cereales mixtos (Potato and mixed cereal snacks) \> SNB \\
                \hspace{1cm}Snacks sin patata (Non-potato snacks) \> SNC \\
    \end{tabbing}

    \vspace{-1.25cm}
    
    \begin{tabbing}
    \hspace{15cm} \= \hspace{0cm} \kill
        \textbf{Sopas, salsas y alimentos varios (Soups, sauces and miscellaneous foods)} \> \textbf{W} \\
            \hspace{0.5cm}Sopas (Soups) \> WA \\
                \hspace{1cm}Sopas caseras (Homemade soups) \> WAA \\
                \hspace{1cm}Sopas enlatadas (Canned soups) \> WAC \\
                \hspace{1cm}Sopas de paquete (Packet soups) \> WAE \\
            \hspace{0.5cm}Salsas (Sauces) \> WC \\
                \hspace{1cm}Salsas lácteas (Dairy sauces) \> WCD \\
                \hspace{1cm}Salsas para ensaladas, aderezos y encurtidos \\
                \hspace{1cm}(Salad sauces, dressings and pickles) \> WCG \\
                \hspace{1cm}Salsas no para ensaladas (Non-salad sauces) \> WCN \\
            \hspace{0.5cm}Encurtidos y chutneys (Pickles and chutneys) \> WE \\
            \hspace{0.5cm}Alimentos varios (Miscellaneous foods) \> WY \\
    \end{tabbing}
\end{small}
\chapter{Apéndice: Código completo}
\label{ch:codigo completo}

\section{funciones_auxiliares.py}



\end{document}
