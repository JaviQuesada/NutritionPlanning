\chapter{Impacto social y medioambiental}
\label{ch:impacto-social-medioambiental}

La realización de una planificación nutricional mediante algoritmos evolutivos tiene un impacto social al influir sobre la salud de las personas. Al optimizar la selección de alimentos, el algoritmo obtiene menús equilibrados según las necesidades de los usuarios. Esto puede ayudar a prevenir enfermedades relacionadas con la mala alimentación, como la obesidad o las enfermedades cardiovasculares. Al ayudar a promover una mejor alimentación, este trabajo se alinea con el \textit{Objetivo de Desarrollo Sostenible} (ODS) 3, llamado \textit{''Salud y bienestar''}, que busca garantizar una vida sana y promover el bienestar para todos en todas las edades.

En cuanto al impacto medioambiental, el conocer la planificación semanal y los alimentos a ingerir ayuda a lograr el ODS 12, \textit{''Producción y consumo responsables''}, que busca asegurar patrones de consumo y producción sostenibles. Es decir, es posible reducir el desperdicio de alimentos y minimizar la huella ecológica que los residuos producen. Pero también, en un futuro se puede personalizar más el algoritmo y la base de datos, como se comenta en la sección \ref{ch:investigacion}, para que, por ejemplo, se puedan elegir los alimentos locales y de temporada, que generan menos contaminación.

Por último, relacionarlo con el ODS 15, \textit{''Vida de ecosistemas terrestres''}, que tiene como objetivo proteger, restaurar y promover el uso sostenible de los ecosistemas terrestres. Al igual que en el ejemplo anterior, se pueden buscar soluciones que tengan un mínimo impacto ambiental. Fomentar a través del algoritmo prácticas amigables y respetuosas con el medio ambiente, como por ejemplo la ganadería extensiva o la pesca sostenible.
