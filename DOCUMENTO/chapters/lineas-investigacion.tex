\chapter{Líneas de investigación}
\label{ch:lineas-investigacion}

Tras realizar este TFG, existen ciertos conceptos en los que no se ha profundizado y que sería interesante desarrollar como ampliaciones futuras.

Para evolucionar el algoritmo y aumentar la calidad nutricional del menú, se pueden realizar varias mejoras. Si bien se han elegido 3 objetivos para simplificar el desarrollo del problema, se pueden incluir más objetivos. Uno de los que se tuvo en cuenta y se descartó es el de la correcta ditribución de calorías a lo largo del día, siguiendo las pautas del Ministerio de Sanidad~\cite{alimentacion_saludable}. Su implementción en el algoritmo sería muy similar al objetivo y restricción de macronutrientes.

Otro aspecto a desarrollar es una mayor personalización del algoritmo en base a las preferencias del individuo. Se pueden considerar razones de religión o de ideología, o incluso la edad del usuario a la hora de aumentar o disminuir la ingesta de un alimento. También la posibilidad de elegir un alimento concreto del listado, más allá de elegir todos los que pertenecen a un mismo grupo.

También a destacar la elección de la base de datos. Se ha seleccionado la base de datos del Reino Unido debido a que es una de las más fáciles a la hora de exportar los datos y y trabajar con ellos, pero presenta tres problemas. El primero es la mezcla de registros entre alimentos ya cocinados y alimentos crudos. Algunos de estos alimentos crudos se pueden ingerir directamente y otros deben pasar previamente por una cocción, por lo que es muy difícil distinguirlos y eliminar los que no se puedan consumir. El segundo problema son los alimentos y platos que aparecen. Se asemejan mucho más a los hábitos de consumo de un ciudadano inglés que al de un español, cuyos hábitos se basan en la dieta mediterránea. El tercero es el idioma, ya que todas las comidas se presentan en inglés.

Tomando de referencia la base de datos \textit{''BEDCA''}(Base de Datos Española de Composición de Alimentos), la creación de una base de datos propia para el problema y en español ayudaría a mejorar el resultado que propone el algoritmo.

Si bien no está señalado dentro de los objetivos del trabajo, se puede desarrollar el software o aplicación que albergue el algoritmo. Aunque se ha hecho un primer acercamiento con la interfaz gráfica propuesta en el apartado \ref{ch:interfaz-grafica}, se puede mejorar su estética y funcionalidad, haciéndola más intuitiva para el usuario.

Por último, destacar las variaciones del algoritmo propuestas para la comparación. Siguiendo la explicación de la sección \ref{ch:marco-teorico}, se pueden añadir más algoritmos multiobjetivo o diferentes manejos de restricciones. Cuanto más opciones se comparen, más optimizado será el resultado que proponga el algoritmo final. Un ejemplo, que no se ha incluido por la falta de eficiencia y el alto tiempo de ejecución, es una función de reparación que busca que todas las restricciones se cumplan antes de evaluar el algoritmo. El código se puede encontrar en el repositorio Github del trabajo~\cite{quesada_reparacion}.